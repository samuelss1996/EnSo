\documentclass[10pt,a4paper]{article}
\usepackage[utf8]{inputenc}
\usepackage[spanish]{babel}
\usepackage{amsmath}
\usepackage{amsfonts}
\usepackage{amssymb}
\usepackage[left=2.5cm,right=2.5cm,top=2.5cm,bottom=2.5cm]{geometry}
\author{Canosa Domínguez, Cristofer \\ Rodríguez Alcaraz, Silvia \\ Seijas Salinas, Orquídea Manuela \\ Soutullo Sobral, Samuel}
\title{Análisis de riegos de la gestión de la configuración en un TFG}

%TODO encabezados y pies de página
\begin{document}
	\maketitle %TODO poner portada bonita
	\newpage
	
	Aquí van las tablas de versiones %TODO
	
	\newpage
	\tableofcontents
	\newpage
	
	\section{Desripción de la práctica}
		TODO %TODO
		
	\section{Descripción del grupo de trabajo}
		TODO %TODO
		
	\section{Planificación de la práctica}
		TODO %TODO meter gantt
		
	\section{Desarrollo de la práctica}
		\subsection{Identificación de activos}
			TODO %TODO
			
		\subsection{Identificación de fuentes de riesgo}
			TODO %TODO
			
		\subsection{Análisis de riesgos}
			\subsubsection{Riesgo R001. Nombre del riesgo}
				\paragraph{Probabilidad} Valor
				\paragraph{Impacto}	Valor
				\paragraph{Descripción} Texto
				\paragraph{Fecha} Valor %Cuando puede producirse el riesgo. Ver el Gantt de Rabenso
				\paragraph{Tratamiento} Valor %Minimización o prevención
				\paragraph{Acción} Texto %Descripción del tratamiento
				\paragraph{Indicadores} Texto %Que nos puede indicar que se está produciendo el riesgo
				\paragraph{Seguimiento}	Texto %Desde cuando empezar a seguir el riesgo. Este apartado solo aparece en riesgos con impactos altos.
	
	\appendix
		\section{Bibliografía y material utilizado}
			TODO %TODO
			
		\section{Relatorio de documentos asociados a éste}
			TODO %TODO
\end{document}